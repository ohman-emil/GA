\selectlanguage{english}
\begin{abstract}
\noindent This work aims to analyse the differences in accuracy between Euler's and Heun's method for solving periodic coupled differential equations. The aforementioned methods are coded in Python, and analytical solutions are obtained using eigenvectors and eigenvalues. Eight periodic systems are examined, of which four are inhomogeneous. It is found that a halving of the step length results in an approximate halving of the total error in Euler's method, if the step length is small enough. In Heun's method, it is found that a halving in step length reduces the total error to about a quarter. This is consistent with previous reseach. Additionally, it is determined that Heun's method is multiple orders of magnitude more accurate compared to Euler's method, if the step length is small enough. Overall, the accuacy in Heun's method is better compared to Euler's method. However, more advanced algorithms are in general more suited towards the computational needs of today.
\end{abstract}
\selectlanguage{swedish}

\noindent\textit{Nyckelord:} Kopplade differentialekvationer, Numeriska lösningsmetoder, Eulers metod, Heuns metod